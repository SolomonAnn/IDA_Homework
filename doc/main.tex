\documentclass[UTF8,12pt]{article} % 12pt 为字号大小
\usepackage{amssymb,amsfonts,amsmath,amsthm}
\usepackage{xeCJK}

\setCJKmainfont[BoldFont={SimHei},ItalicFont={KaiTi}]{SimSun}
\setCJKsansfont{SimHei}
\setCJKfamilyfont{zhsong}{SimSun}
\setCJKfamilyfont{zhhei}{SimHei}
\setCJKfamilyfont{zhkai}{KaiTi}
\setCJKfamilyfont{zhfs}{FangSong}
\setCJKfamilyfont{zhli}{LiSu}
\setCJKfamilyfont{zhyou}{YouYuan}

\newcommand*{\songti}{\CJKfamily{zhsong}} % 宋体
\newcommand*{\heiti}{\CJKfamily{zhhei}}   % 黑体
\newcommand*{\kaiti}{\CJKfamily{zhkai}}  % 楷体
\newcommand*{\fangsong}{\CJKfamily{zhfs}} % 仿宋
\newcommand*{\lishu}{\CJKfamily{zhli}}    % 隶书
\newcommand*{\yuanti}{\CJKfamily{zhyou}} % 圆体

\usepackage{indentfirst}
\setlength{\parindent}{2em}

\renewcommand{\baselinestretch}{1.4}

\usepackage[a4paper]{geometry}
\geometry{verbose,
  tmargin=2cm,
  bmargin=2cm,
  lmargin=3cm,
  rmargin=3cm
}

\usepackage[x11names]{xcolor} % must before tikz, x11names defines RoyalBlue3
\usepackage{graphicx}
\usepackage{pstricks,pst-plot,pst-eps}
\usepackage{subfig}
\def\pgfsysdriver{pgfsys-dvipdfmx.def} % put before tikz
\usepackage{tikz}
\usepackage{verbatim}
\usepackage{url}
\usepackage{enumerate}
\usepackage{diagbox}

\begin{document}

\title{IDA课程作业实验报告}
\author{软件51$\ \ $  2015013190$\ \ $  安彦哲}
\date{2018.11.22}
\maketitle

\section{数据预处理}
\subsection{遗漏数据处理}
\subsubsection{$Race\ (2\%)$}
\begin{enumerate}[1)]
  \item 经过统计,所给数据表中共有 $101766$ 条数据,其中 $AfricanAmerican$ 有 $19210$ 条,$Asian$ 有 $641$ 条,
        $Caucasian$ 有 $76099$ 条, $Hispanic$ 有 $2037$ 条, $Other$ 有 $1506$ 条,遗漏 $2273$ 条。
  \item 处理方案:将遗漏处均填补为 $Caucasian$ (众数)。
\end{enumerate}
\subsubsection{$Weight\ (97\%)$}
\begin{enumerate}[1)]
  \item 统计不同 $Age$ ,不同 $Gender$ ,不同 $Weight$ 的数据条数,得到的结果如下: \\
    \begin{tabular}{|c|c|c|c|c|c|}
      \hline
      \diagbox{Weight}{Male/Female}{Age} & [0-10) & [10-20) & [20-30) & [30-40) & [40-50) \\
      \hline
      [0-25) & $3/0$ & $0/0$ & $0/1$ & $1/0$ & $5/2$\\
      \hline
      [25-50) & $1/1$ & $1/3$ & $0/1$ & $1/3$ & $3/4$ \\
      \hline
      [50-75) & $0/0$ & $4/10$ & $16/23$ & $12/12$ & $18/20$\\
      \hline
      [75-100) & $0/0$ & $0/0$ & $7/10$ & $11/23$ & $44/36$ \\
      \hline
      [100-125) & $0/0$ & $0/0$ & $2/7$ & $6/4$ & $30/36$\\
      \hline
      [125-150) & $0/0$ & $0/0$ & $0/1$ & $6/3$ & $15/10$ \\
      \hline
      [150-175) & $0/0$ & $0/0$ & $1/0$ & $0/2$ & $2/2$\\
      \hline
      [175-200) & $0/0$ & $0/0$ & $0/0$ & $0/0$ & $3/1$ \\
      \hline
      >200 & $0/0$ & $0/0$ & $0/0$ & $0/1$ & $0/0$\\
      \hline
    \end{tabular} \\
    \begin{tabular}{|c|c|c|c|c|c|}
      \hline
      \diagbox{Weight}{Male/Female}{Age} & [50-60) & [60-70) & [70-80) & [80-90) & [90-100) \\
      \hline
      [0-25) & $3/1$ & $5/7$ & $7/7$ & $2/4$ & $0/0$\\
      \hline
      [25-50) & $6/7$ & $4/9$ & $6/9$ & $2/27$ & $2/7$ \\
      \hline
      [50-75) & $37/59$ & $67/88$ & $71/170$ & $66/165$ & $11/48$\\
      \hline
      [75-100) & $88/100$ & $150/145$ & $252/215$ & $136/95$ & $15/9$ \\
      \hline
      [100-125) & $78/60$ & $123/70$ & $119/55$ & $25/10$ & $0/0$\\
      \hline
      [125-150) & $25/20$ & $22/14$ & $13/9$ & $4/3$ & $0/0$ \\
      \hline
      [150-175) & $4/7$ & $9/4$ & $1/3$ & $0/0$ & $0/0$\\
      \hline
      [175-200) & $2/3$ & $0/1$ & $1/0$ & $0/0$ & $0/0$ \\
      \hline
      >200 & $0/0$ & $1/1$ & $0/0$ & $0/0$ & $0/0$\\
      \hline
    \end{tabular}
  \item 根据以上数据计算出不同 $Age$ ,不同 $Gender$ 的平均 $Weight$ : \\
    \begin{tabular}{|c|c|c|c|c|c|}
      \hline
      \diagbox{Gender}{Weight}{Age} & [0-10) & [10-20) & [20-30) & [30-40) & [40-50) \\
      \hline
      Male & $[0-25)$ & $[50-75)$ & $[75-100)$ & $[75-100)$ & $[75-100)$\\
      \hline
      Female & $[25-50)$ & $[50-75)$ & $[75-100)$ & $[75-100)$ & $[75-100)$ \\
      \hline
      \diagbox{Gender}{Weight}{Age} & [50-60) & [60-70) & [70-80) & [80-90) & [90-100) \\
      \hline
      Male & $[75-100)$ & $[75-100)$ & $[75-100)$ & $[75-100)$ & $[50-75)$\\
      \hline
      Female & $[75-100)$ & $[75-100)$ & $[75-100)$ & $[50-75)$ & $[50-75)$ \\
      \hline
    \end{tabular} \\
    并用平均值填补缺漏值。
  \item 因为 $Unknown/Invalid$ 记录条数较少( $3$ 条,两条 $[70-80)$ ,一条 $[60-70)$ ),所以选择手动处理(根据这两个年龄段的 $Weight$ 平均值,填补为 $[75-100)$)。
\end{enumerate}
\section{}

\section{}

\end{document}